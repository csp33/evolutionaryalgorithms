\documentclass[12pt,english]{article}
\usepackage[english]{babel}
\usepackage{graphicx}
\usepackage{amsmath}
\usepackage{multirow}
\usepackage{pgfplots}
\usepackage{subcaption}
\usepackage{amssymb}
\usepackage[hidelinks]{hyperref}
\usepackage{caption}
\usepackage{amsthm}
\usepackage{multicol}
\pgfplotsset{compat=1.16}
\usepackage{minted}
\usepackage{float}
\usepackage{titling}
\usepackage{soul}
\usepackage{listings}
\usepackage{array}
\graphicspath{ {../img/}}
\selectlanguage{english}
\usepackage[nottoc]{tocbibind}
\usepackage[utf8]{inputenc}
\usepackage{graphicx}
\usepackage[a4paper,left=2cm,right=2cm,top=2.5cm,bottom=2.5cm]{geometry}
\RecustomVerbatimEnvironment{Verbatim}{BVerbatim}{}


\title{Evolutionary Algorithms}
\setlength{\droptitle}{10em}
\author{Carlos Sánchez Páez}

\makeindex
\begin{document}


\begin{titlepage}

\newlength{\centeroffset}
\setlength{\centeroffset}{-0.5\oddsidemargin}
\addtolength{\centeroffset}{0.5\evensidemargin}
\thispagestyle{empty}

\noindent\hspace*{\centeroffset}
\begin{minipage}{\textwidth}

\centering
\includegraphics[width=0.75\textwidth]{bme_logo.jpg}\\[1.4cm]

\textsc{ \Large Evolutionary Algorithms\\[4cm]}

\textsc{\Huge Homework}\\[0.75cm]

{\Large\bfseries Second task\\}
\end{minipage}

\vspace{8cm}
\noindent\hspace*{\centeroffset}
\begin{minipage}{\textwidth}
\centering

\textbf{Author}\\ {Carlos Sánchez Páez}\\
\texttt{http://www.github.com/csp98}\\[0.5cm]
\textsc{Budapest University of Technology and Economics}\\
\vspace{1cm}
\textsc{Academic year 2018-2019}
\end{minipage}
\end{titlepage}
\thispagestyle{empty}

\newpage


\begin{enumerate}
	\item Let us suppose, that an algorithm’s running time is polynomial, that is $cn^\alpha$ for some $\alpha$, $c \in \mathbb{R}$ constants. Give an estimate for
$c$ and $\alpha$ if for input lengths n = [4, 5, 6, 7, 8, 9, 10] we measured the following running times [37.1 58.7 84.0 115.1 150.8 190.9 235.2].
	\item Let us suppose, that we have a population containing 4 individuals called e1, e2, e3, e4. Their fitness’s are 0.4, 0.7, 0.3, 0.05. We use a roulette-wheel selection to select the four parents.
	\begin{itemize}
		\item What is the probability, that e2 won’t be chosen as parent at all?
		\item What is the probability, that e3 will be chosen two times?
	\end{itemize}
	\item What can be a good measure of performance for a genetic algorithm? Justify your answer! Using your measure find the optimal probability of the mutation for the backpacking problem, using elitism, a tournament selection with k = 4, a fitness function described in the first lecture (sum of the values if the sum of the weights is below or equal to the capacity, 0 otherwise). Is there a significant difference in the efficiency between the optimal parameter and setting the probability of mutation to 0?

\end{enumerate}


\begin{thebibliography}{9}

\bibitem{Course Webpage}
Course Webpage
\\\texttt{http://math.bme.hu/~safaro/evolalgen.html}


\bibitem{Webpage4}
\texttt{https://tex.stackexchange.com/}

\end{thebibliography}


\end{document}
