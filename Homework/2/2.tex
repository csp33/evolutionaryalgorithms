\documentclass[12pt,english]{article}
\usepackage[english]{babel}
\usepackage{graphicx}
\usepackage{amsmath}
\usepackage{multirow}
\usepackage{pgfplots}
\usepackage{subcaption}
\usepackage{amssymb}
\usepackage[hidelinks]{hyperref}
\usepackage{caption}
\usepackage{amsthm}
\usepackage{multicol}
\pgfplotsset{compat=1.16}
\usepackage{minted}
\usepackage{float}
\usepackage{titling}
\usepackage{soul}
\usepackage{listings}
\newenvironment{statement}{\fontfamily{ptm}\selectfont}{\par}
\usepackage{array}
\graphicspath{ {../img/}}
\selectlanguage{english}
\usepackage[nottoc]{tocbibind}
\usepackage[utf8]{inputenc}
\usepackage{graphicx}
\usepackage[a4paper,left=2cm,right=2cm,top=2.5cm,bottom=2.5cm]{geometry}
\RecustomVerbatimEnvironment{Verbatim}{BVerbatim}{}


\title{Evolutionary Algorithms}
\setlength{\droptitle}{10em}
\author{Carlos Sánchez Páez}

\makeindex
\begin{document}


\begin{titlepage}

\newlength{\centeroffset}
\setlength{\centeroffset}{-0.5\oddsidemargin}
\addtolength{\centeroffset}{0.5\evensidemargin}
\thispagestyle{empty}

\noindent\hspace*{\centeroffset}
\begin{minipage}{\textwidth}

\centering
\includegraphics[width=0.75\textwidth]{bme_logo.jpg}\\[1.4cm]

\textsc{ \Large Evolutionary Algorithms\\[4cm]}

\textsc{\Huge Homework}\\[0.75cm]

{\Large\bfseries Second task\\}
\end{minipage}

\vspace{8cm}
\noindent\hspace*{\centeroffset}
\begin{minipage}{\textwidth}
\centering

\textbf{Author}\\ {Carlos Sánchez Páez}\\
\texttt{http://www.github.com/csp98}\\[0.5cm]
\textsc{Budapest University of Technology and Economics}\\
\vspace{1cm}
\textsc{Academic year 2018-2019}
\end{minipage}
\end{titlepage}
\thispagestyle{empty}

\newpage


\begin{enumerate}

	\item
		\begin{statement}
			Let us suppose, that an algorithm’s running time is polynomial, that is $cn^\alpha$ for some $\alpha$, $c \in \mathbb{R}$ constants. Give an estimate for
			$c$ and $\alpha$ if for input lengths n = [4, 5, 6, 7, 8, 9, 10] we measured the following running times [37.1 58.7 84.0 115.1 150.8 190.9 235.2].
		\end{statement}

		We will solve the exercise with the help of \textsc{gnuplot}. First, we will define the function and then tell the program to fit it to the given values:
		\begin{figure}[H]
			\centering
			\begin{minted}[gobble=4]{gnuplot}
				f(x) = c * x ** a
				fit f(x) 'ex1_data.txt' via a,c
			\end{minted}
		\end{figure}
		The final set of parameters obtained is the following:
		\begin{figure}[H]
			\centering
			\begin{tabular}{|c|c|c|}
				\hline
				 & $a$ & $c$ \\
				 \hline
				 \textbf{Value} & $2.00895$ & $2.3073$ \\
				 \hline
				 \textbf{Error} & $0.25\%$ & $1.088\%$ \\
				 \hline
			\end{tabular}
		\end{figure}

		So our function is $f(n) = 2.3073n^{2.00895}$.

	\item
		\begin{statement}
			Let us suppose, that we have a population containing 4 individuals called $e_1$, $e_2$, $e_3$, $e_4$. Their fitness’s are $0.4$, $0.7$, $0.3$, $0.05$. We use a roulette-wheel selection to select the four parents. Answer these two questions for $f(e)$ and also for the scaled $\hat{f}(e)=f^2(e)$ fitness function.
		\end{statement}
	\begin{figure}[H]
		\centering
		\begin{tabular}{|c|c|c|}
			\hline
		 	& $f(e_i)$ & $f^{2}(e_i)$ \\
		 	\hline
			$e_1$ & $0.4$ & $0.16$ \\
			\hline
			$e_2$ & $0.7$ & $0.49$ \\
			\hline
			$e_3$ & $0.3$ & $0.09$ \\
			\hline
			$e_4$ & $0.05$ & $0.0025$ \\
			\hline
			\textbf{Sum} & $1.45$ & $0.7425$ \\
			\hline
		\end{tabular}
	\end{figure}
	\begin{itemize}
		\item
		\begin{statement}
			What is the probability, that $e_2$ won’t be chosen as parent at all?
		\end{statement}

			To solve this question we will use the complement rule of probability. If $P(e_2)$ is the probability of electing $e_2$, $P(\overline{e_2}) = 1 - P(e_2)$ is the probabilty of not choosing it at all. We will start calculating $P(e_2)$.
			\[
				P(e_2) = \frac{f(e_2)}{\sum_{i=1}^{n}{f(e_i)}} = \frac{0.7}{1.45} = 0.482758621
			\]
			\begin{center}
			$P(\overline{e_2}) = 1 - P(e_2) = 1 - 0.482758621 = 0.517241379 $\\
			\end{center}

			So the probability is \textbf{51.72\%}.

			For $f^2(e)$ we apply the same procedure:

			\[
				P^{\prime}(e_2) = \frac{f^{2}(e_2)}{\sum_{i=1}^{n}{f^{2}(e_i)}} = \frac{0.49}{0.7425} = 0.65993266
			\]
			\begin{center}
			$P^{\prime}(\overline{e_2}) = 1 - P^{\prime}(e_2) = 1 - 0.65993266 = 0.34006734 $\\
			\end{center}

			So the probability is \textbf{34.01\%}.

		\item
		\begin{statement}
			What is the probability, that $e_3$ will be chosen two times?
		\end{statement}

		The probability of the intersection is defined by the following statement:
		$ P(A \cap B) = P(A) \cdot P(B/A)$.
		$P(B/A)$ means the probability of $B$ knowing that $A$ happened. In this case, the selection of the parents is made in an \emph{equiprobable} way, so $P(B/A) = P(B)$.
		\[
			P(e_3 \cap e_3) = (P(e_3))^2 = (\frac{f(e_3)}{\sum_{i=1}^{n}{f(e_i)}})^2 = (\frac{0.3}{1.45})^2 =
			0.206896552^2= 0.042806183
		\]
		So the probability is \textbf{4.28\%}.

		Now, let's calculate it for $f^2(e)$:
		\[
			P^{\prime}(e_3 \cap e_3) = (P^{\prime}(e_3))^2 = (\frac{f^{2}(e_3)}{\sum_{i=1}^{n}{f^{2}(e_i)}})^2 = (\frac{0.09}{0.7425})^2 =
			0.121212121^2= 0.014692378
		\]
		So the probability is \textbf{1.47\%}.

	\end{itemize}
	\item
		\begin{statement}		
			What can be a good measure of performance for a genetic algorithm? Justify your answer! Using your measure find the optimal probability of the mutation for the backpacking problem, using elitism, a tournament selection with k = 4, a fitness function described in the first lecture (sum of the values if the sum of the weights is below or equal to the capacity, 0 otherwise). Is there a significant difference in the efficiency between the optimal parameter and setting the probability of mutation to 0?
		\end{statement}

	A good measure of performance for a genetic algorithm can be the approximation that it makes to the real optimum solution. The lower the difference is, the best performance the algorithm gives.

\end{enumerate}


\begin{thebibliography}{9}

\bibitem{Course Webpage}
Course Webpage
\\\texttt{http://math.bme.hu/~safaro/evolalgen.html}


\bibitem{Webpage4}
\texttt{https://tex.stackexchange.com/}

\bibitem{Webpage5}
\texttt{https://stattrek.com/probability/probability-rules.aspx}


\end{thebibliography}


\end{document}
