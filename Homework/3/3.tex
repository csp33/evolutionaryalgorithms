\documentclass[12pt,english]{article}
\usepackage[english]{babel}
\usepackage{graphicx}
\usepackage{amsmath}
\usepackage{multirow}
\usepackage{pgfplots}
\usepackage{subcaption}
\usepackage{amssymb}
\usepackage[hidelinks]{hyperref}
\usepackage{caption}
\usepackage{amsthm}
\usepackage{multicol}
\pgfplotsset{compat=1.16}
\usepackage{minted}
\usepackage{float}
\usepackage{titling}
\usepackage{soul}
\usepackage{listings}
\newenvironment{statement}{\fontfamily{ptm}\selectfont}{\par}
\usepackage{array}
\graphicspath{ {../img/}}
\selectlanguage{english}
\usepackage[nottoc]{tocbibind}
\usepackage[utf8]{inputenc}
\usepackage{graphicx}
\usepackage[a4paper,left=2cm,right=2cm,top=2.5cm,bottom=2.5cm]{geometry}
\RecustomVerbatimEnvironment{Verbatim}{BVerbatim}{}


\title{Evolutionary Algorithms}
\setlength{\droptitle}{10em}
\author{Carlos Sánchez Páez}

\makeindex
\begin{document}


\begin{titlepage}

\newlength{\centeroffset}
\setlength{\centeroffset}{-0.5\oddsidemargin}
\addtolength{\centeroffset}{0.5\evensidemargin}
\thispagestyle{empty}

\noindent\hspace*{\centeroffset}
\begin{minipage}{\textwidth}

\centering
\includegraphics[width=0.75\textwidth]{bme_logo.jpg}\\[1.4cm]

\textsc{ \Large Evolutionary Algorithms\\[4cm]}

\textsc{\Huge Homework}\\[0.75cm]

{\Large\bfseries Third task\\}
\end{minipage}

\vspace{8cm}
\noindent\hspace*{\centeroffset}
\begin{minipage}{\textwidth}
\centering

\textbf{Author}\\ {Carlos Sánchez Páez}\\
\texttt{http://www.github.com/csp98}\\[0.5cm]
\textsc{Budapest University of Technology and Economics}\\
\vspace{1cm}
\textsc{Academic year 2018-2019}
\end{minipage}
\end{titlepage}
\thispagestyle{empty}

\newpage


\begin{enumerate}

	\item
		\begin{statement}
    Let us suppose, that the potential solutions are represented with 4-length 0-1 sequences. Construct a fitness function, for which the global optimum is in 0000, but every scheme has the property, that if we change every 0 to 1 the fitness of the scheme increases. Prove this property for your fitness function.
		\end{statement}


	\item
		\begin{statement}
    Consider the following statement: "If the Gray codes of two numbers only differ in one position, then the distance of the two numbers is 1." Prove this statement if it’s true, or give a counterexample if it isn’t.
		\end{statement}

    The statement is true. Let's prove it:\\
    Let $n \in \mathbb{N}$ and $n+1$ its consecutive.\\
		$n)_{10}=b_1b_2...b_k)_2=b_1(b_1 \oplus b_2)...(b_{k-1} \oplus b_k))_{Gray}$\\
		$n+1)_{10} = b_1b_2...b_k)_2 + 00...001)_2 $

		If the binary representation of $n$ ends in $1$, $n+1)_2$ will shift $1$ to the left. If it ends in $0$, the unique change will be the last bit ($0 \rightarrowtail 1$).\\

		So, as we need to do the $XOR$ operation between the bits (XOR means that the result will be $1$ only if the bits operated are disctinct) the difference between $n)_{Gray}$ and $n+1)_{Gray}$ will be one bit (the one that has been shifted or not). Let us see an example:

		$11_{10} = 1011)_2 = 1100)_{Gray}$\\
		$12_{10} = 1100)_2 = 1110)_{Gray}$\\


	\item
		\begin{statement}
    Let us investigate the genetic algorithm for the backpacking problem. Your task is to count how many schemes the algorithm evaluates per generation. (You can use your own program or modify the one on the homepage).
		\end{statement}

		Holland proved that a genetic algorithm evaluates O($\mu^3$) schemes in one generation, where $\mu$ represents the number of individuals in the population.

		So, the solution for the problem is $30^3 = 27000$


\end{enumerate}


\begin{thebibliography}{9}

\bibitem{Course Webpage}
Course Webpage
\\\texttt{http://math.bme.hu/~safaro/evolalgen.html}


\bibitem{Webpage4}
\texttt{https://tex.stackexchange.com/}


\end{thebibliography}


\end{document}
