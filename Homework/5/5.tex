\documentclass[12pt,english]{article}
\usepackage[english]{babel}
\usepackage{graphicx}
\usepackage{amsmath}
\usepackage{multirow}
\usepackage{pgfplots}
\usepackage{subcaption}
\usepackage{amssymb}
\usepackage[hidelinks]{hyperref}
\usepackage{caption}
\usepackage{amsthm}
\usepackage{multicol}
\pgfplotsset{compat=1.16}
\usepackage{minted}
\usepackage{float}
\usepackage{titling}
\usepackage{soul}
\usepackage{listings}
\newenvironment{statement}{\fontfamily{ptm}\selectfont}{\par}
\usepackage{array}
\graphicspath{ {../img/}}
\selectlanguage{english}
\usepackage[nottoc]{tocbibind}
\usepackage[utf8]{inputenc}
\usepackage{graphicx}
\usepackage[a4paper,left=2cm,right=2cm,top=2.5cm,bottom=2.5cm]{geometry}
\RecustomVerbatimEnvironment{Verbatim}{BVerbatim}{}


\title{Evolutionary Algorithms}
\setlength{\droptitle}{10em}
\author{Carlos Sánchez Páez}

\makeindex
\begin{document}


\begin{titlepage}

\newlength{\centeroffset}
\setlength{\centeroffset}{-0.5\oddsidemargin}
\addtolength{\centeroffset}{0.5\evensidemargin}
\thispagestyle{empty}

\noindent\hspace*{\centeroffset}
\begin{minipage}{\textwidth}

\centering
\includegraphics[width=0.75\textwidth]{bme_logo.jpg}\\[1.4cm]

\textsc{ \Large Evolutionary Algorithms\\[4cm]}

\textsc{\Huge Homework}\\[0.75cm]

{\Large\bfseries Fifth task\\}
\end{minipage}

\vspace{8cm}
\noindent\hspace*{\centeroffset}
\begin{minipage}{\textwidth}
\centering

\textbf{Author}\\ {Carlos Sánchez Páez}\\
\texttt{http://www.github.com/csp98}\\[0.5cm]
\textsc{Budapest University of Technology and Economics}\\
\vspace{1cm}
\textsc{Academic year 2018-2019}
\end{minipage}
\end{titlepage}
\thispagestyle{empty}

\newpage


\begin{enumerate}

	\item
		\begin{statement}
		Prove the following statement: \emph{If we can find a Hamiltonian cycle in a digraph in polynomial time, then using this algorithm we can find a Hamiltonian cycle in a graph.}.
		\end{statement}


	\item
		\begin{statement}
		Let us suppose, that we use one of the four crossover operator (CX, OX, PMX, EX) for 9-long permutation pairs, the matching segment is the 4-7. positions if there is any. Is it possible, that that the two parents aren’t identical, but the offspring is identical to one of the parents?
		\end{statement}

	\item
		\begin{statement}
		Consider 9-long permutations. Applying CX operator, the permutation pair is divided to cycles. Count the number of cycles for 106 random permutation pairs and make a histogram of the distribution of the number of cycles.
		\end{statement}



\end{enumerate}


\begin{thebibliography}{9}

\bibitem{Course Webpage}
Course Webpage
\\\texttt{http://math.bme.hu/~safaro/evolalgen.html}


\bibitem{Webpage4}
\texttt{https://tex.stackexchange.com/}


\end{thebibliography}


\end{document}
