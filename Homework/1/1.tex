\documentclass[12pt,english]{article}
\usepackage[english]{babel}
\usepackage{graphicx}
\usepackage{amsmath}
\usepackage{multirow}
\usepackage{subcaption}
\usepackage{amssymb}
\usepackage[hidelinks]{hyperref}
\usepackage{caption}
\usepackage{amsthm}
\usepackage{multicol}
\usepackage[outputdir=build]{minted}
\usepackage{float}
\usepackage{titling}
\usepackage{soul}
\usepackage{listings}
\usepackage{array}
\graphicspath{ {../img/}}
\selectlanguage{english}
\usepackage[nottoc]{tocbibind}
\usepackage[utf8]{inputenc}
\usepackage{graphicx}
\usepackage[a4paper,left=2cm,right=2cm,top=2.5cm,bottom=2.5cm]{geometry}
\RecustomVerbatimEnvironment{Verbatim}{BVerbatim}{}


\title{Evolutionary Algorithms}
\setlength{\droptitle}{10em}
\author{Carlos Sánchez Páez}

\makeindex
\begin{document}


\begin{titlepage}

\newlength{\centeroffset}
\setlength{\centeroffset}{-0.5\oddsidemargin}
\addtolength{\centeroffset}{0.5\evensidemargin}
\thispagestyle{empty}

\noindent\hspace*{\centeroffset}
\begin{minipage}{\textwidth}

\centering
\includegraphics[width=0.75\textwidth]{bme_logo.jpg}\\[1.4cm]

\textsc{ \Large Evolutionary Algorithms\\[4cm]}

-\textsc{\Huge Homework}\\[0.75cm]

{\Large\bfseries First task\\}
\end{minipage}

\vspace{8cm}
\noindent\hspace*{\centeroffset}
\begin{minipage}{\textwidth}
\centering

\textbf{Author}\\ {Carlos Sánchez Páez}\\
\texttt{http://www.github.com/csp98}\\[0.5cm]
\textsc{Budapest University of Technology and Economics}\\
\vspace{1cm}
\textsc{Academic year 2018-2019}
\end{minipage}
\end{titlepage}
\thispagestyle{empty}

\newpage


\begin{enumerate}
	\item Let us suppose, that the run-time of an algorithm is monotone increasing in n, which is the length of the input. Investigate the following statements:
	\begin{enumerate}
		\item If we plot the run-time on log-log scale than we obtain a monoton increasing function.
		\item If the run-time is not only monotone increasing, but also convex, then the plot on log-log scale will be convex.
	\end{enumerate}
	Prove the true statement(s) and give counterexample(s) to the false one(s).
	\item Let us suppose, that two possible solution are coded with 0101010 and 0001000. If they are the parents can any of their offspring be:
	\begin{itemize}
		\item 0101010
		\item 1111111
		\item 0000000
	\end{itemize}
in case we use:
\begin{itemize}
	\item onepoint-crossover ?
	\item multiplepoints-crossover ?
	\item uniform-crossover ?
\end{itemize}

\begin{figure}[H]
\centering
	\begin{tabular}{|m{3cm}|m{4cm}|m{4cm}|m{4cm}|}
	\hline
		\textbf{Offspring} & \textbf{Onepoint-crossover} & \textbf{Multiplepoints-crossover} & \textbf{Uniform-crossover}\\
		\hline
		 0101010 & Yes, if we put the breakpoint in the first position (0\textbf{101010} + \textbf{0}001000)& & \\
		 \hline
		 1111111 & & & \\
		 \hline
		 0000000 & & & \\
		 \hline
	\end{tabular}
\end{figure}

\item How can we represent with a fixed length 0-1 series the solutions of the following problem?

We have an n × n sized grid, and on each edge a real positive number.
We are looking for a path from the upper left corner point to the lower
right point, which goes only to the right or down, where the sum of the numbers along the path is minimal. Write a program (using your favorite programming language) to solve this problem using a suitable genetic algorithm.


\end{enumerate}


\begin{thebibliography}{9}

\bibitem{Course Webpage}
Course Webpage
\\\texttt{http://math.bme.hu/~safaro/evolalgen.html}

\end{thebibliography}



\end{document}
