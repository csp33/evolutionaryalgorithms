\documentclass[12pt,english]{article}
\usepackage[english]{babel}
\usepackage{graphicx}
\usepackage{amsmath}
\usepackage{multirow}
\usepackage{subcaption}
\usepackage{amssymb}
\usepackage[hidelinks]{hyperref}
\usepackage{caption}
\usepackage{amsthm}
\usepackage{multicol}
\usepackage{minted}
\usepackage{float}
\usepackage{titling}
\usepackage{soul}
\usepackage{listings}
\usepackage{array}
\graphicspath{ {../img/}}
\selectlanguage{english}
\usepackage[nottoc]{tocbibind}
\usepackage[utf8]{inputenc}
\usepackage{graphicx}
\usepackage[a4paper,left=2cm,right=2cm,top=2.5cm,bottom=2.5cm]{geometry}
\RecustomVerbatimEnvironment{Verbatim}{BVerbatim}{}


\title{Evolutionary Algorithms}
\setlength{\droptitle}{10em}
\author{Carlos Sánchez Páez}

\makeindex
\begin{document}


\begin{titlepage}

\newlength{\centeroffset}
\setlength{\centeroffset}{-0.5\oddsidemargin}
\addtolength{\centeroffset}{0.5\evensidemargin}
\thispagestyle{empty}

\noindent\hspace*{\centeroffset}
\begin{minipage}{\textwidth}

\centering
\includegraphics[width=0.75\textwidth]{bme_logo.jpg}\\[1.4cm]

\textsc{ \Large Evolutionary Algorithms\\[4cm]}

-\textsc{\Huge Homework}\\[0.75cm]

{\Large\bfseries First task\\}
\end{minipage}

\vspace{8cm}
\noindent\hspace*{\centeroffset}
\begin{minipage}{\textwidth}
\centering

\textbf{Author}\\ {Carlos Sánchez Páez}\\
\texttt{http://www.github.com/csp98}\\[0.5cm]
\textsc{Budapest University of Technology and Economics}\\
\vspace{1cm}
\textsc{Academic year 2018-2019}
\end{minipage}
\end{titlepage}
\thispagestyle{empty}

\newpage


\begin{enumerate}
	\item Let us suppose, that the run-time of an algorithm is monotone increasing in n, which is the length of the input. Investigate the following statements:
	\begin{enumerate}
		\item If we plot the run-time on log-log scale than we obtain a monoton increasing function.
		\item If the run-time is not only monotone increasing, but also convex, then the plot on log-log scale will be convex.
	\end{enumerate}
	Prove the true statement(s) and give counterexample(s) to the false one(s).
	\item Let us suppose, that two possible solution are coded with 0101010 and 0001000. If they are the parents can any of their offspring be:
	\begin{itemize}
		\item 0101010
		\item 1111111
		\item 0000000
	\end{itemize}
in case we use:
\begin{itemize}
	\item onepoint-crossover ?
	\item multiplepoints-crossover ?
	\item uniform-crossover ?
\end{itemize}

\begin{figure}[H]
\centering
	\begin{tabular}{|m{3cm}|m{4cm}|m{4cm}|m{4cm}|}
	\hline
		\textbf{Offspring} & \textbf{Onepoint-crossover} & \textbf{Multiplepoints-crossover} & \textbf{Uniform-crossover}\\
		\hline
		 0101010 & Yes, if we put the breakpoint in the first position (0\textbf{101010} + \textbf{0}001000)& Yes. If we put the breakpoints in the first and last position (0\textbf{10101}0 + \textbf{0}00100\textbf{0}) & Yes, if the probability chooses the right parent in the right moment. Example: 0\textbf{1}0\textbf{1}0\textbf{1}0 + \textbf{0}0\textbf{0}1\textbf{0}0\textbf{0} \\
		 \hline
		 1111111 & No. In many positions there are not ones in the parents.& No. In many positions there are not ones in the parents.& No. In many positions there are not ones in the parents.\\
		 \hline
		 0000000 & No. In some positions there are not zeros in the parents. & No, because in the $4^{th}$ position both parents have a one.& No, because in the $4^{th}$ position both parents have a one.\\
		 \hline
	\end{tabular}
\end{figure}

\item How can we represent with a fixed length 0-1 series the solutions of the following problem?

We have an n × n sized grid, and on each edge a real positive number.
We are looking for a path from the upper left corner point to the lower
right point, which goes only to the right or down, where the sum of the numbers along the path is minimal. Write a program (using your favorite programming language) to solve this problem using a suitable genetic algorithm.

The representation of the problem will be the following:
\begin{itemize}
	\item \textbf{0} = Move right
	\item \textbf{1} = Move down
\end{itemize}
Each individual will have a genotype (size n + 2) which indicates the steps that have to be completed to reach the destination. The fitness funcion is the sum of the costs of the points crossed during the path (it has to be minimized).\\

In my algorithm I use the onepoint-crossover technique for the reproduction. The population size is 10 and the mutation chance, 20\%.

\end{enumerate}


\begin{thebibliography}{9}

\bibitem{Course Webpage}
Course Webpage
\\\texttt{http://math.bme.hu/~safaro/evolalgen.html}

\bibitem{Webpage}
\texttt{http://www.rubicite.com/Tutorials/GeneticAlgorithms.aspx}

\bibitem{Webpage2}
\texttt{https://towardsdatascience.com/genetic-algorithm-implementation-in-python-5ab67bb124a6}

\bibitem{Webpage3}
\texttt{https://docs.python.org/3/}


\bibitem{Webpage4}
\texttt{https://tex.stackexchange.com/}




\end{thebibliography}



\end{document}
