\documentclass[12pt,english]{article}
\usepackage[english]{babel}
\usepackage{graphicx}
\usepackage{amsmath}
\usepackage{multirow}
\usepackage{pgfplots}
\usepackage{subcaption}
\usepackage{amssymb}
\usepackage[hidelinks]{hyperref}
\usepackage{caption}
\usepackage{amsthm}
\usepackage{multicol}
\pgfplotsset{compat=1.16}
\usepackage{minted}
\usepackage{float}
\usepackage{titling}
\usepackage{soul}
\usepackage{listings}
\newenvironment{statement}{\fontfamily{ptm}\selectfont}{\par}
\usepackage{array}
\graphicspath{ {../img/}}
\selectlanguage{english}
\usepackage[nottoc]{tocbibind}
\usepackage[utf8]{inputenc}
\usepackage{graphicx}
\usepackage[a4paper,left=2cm,right=2cm,top=2.5cm,bottom=2.5cm]{geometry}
\RecustomVerbatimEnvironment{Verbatim}{BVerbatim}{}


\title{Evolutionary Algorithms}
\setlength{\droptitle}{10em}
\author{Carlos Sánchez Páez}

\makeindex
\begin{document}


\begin{titlepage}

\newlength{\centeroffset}
\setlength{\centeroffset}{-0.5\oddsidemargin}
\addtolength{\centeroffset}{0.5\evensidemargin}
\thispagestyle{empty}

\noindent\hspace*{\centeroffset}
\begin{minipage}{\textwidth}

\centering
\includegraphics[width=0.75\textwidth]{bme_logo.jpg}\\[1.4cm]

\textsc{ \Large Evolutionary Algorithms\\[4cm]}

\textsc{\Huge Homework}\\[0.75cm]

{\Large\bfseries Sixth task\\}
\end{minipage}

\vspace{8cm}
\noindent\hspace*{\centeroffset}
\begin{minipage}{\textwidth}
\centering

\textbf{Author}\\ {Carlos Sánchez Páez}\\
\texttt{http://www.github.com/csp98}\\[0.5cm]
\textsc{Budapest University of Technology and Economics}\\
\vspace{1cm}
\textsc{Academic year 2018-2019}
\end{minipage}
\end{titlepage}
\thispagestyle{empty}

\newpage


\begin{enumerate}

	\item
		\begin{statement}
		Calculate the probability, that if we choose two random permutation as parents, then the CX operator produces
			\begin{itemize}
				\item 1 cycle
				\item $n$ cycles, where $n$ is the length of the permutations.
			\end{itemize}
		\end{statement}

	\item
		\begin{statement}
		Is it possible for any of the four crossover operators (PMX,EX, CX, OX), that if we exchange the order of the two, not identical parents, that the offspring are identical?
		\end{statement}
		In the last task we proved that we could get an identical offspring with all these operators. \\
		As \textbf{EX} is the only operator which does not care about the order of the parents, is the only one which can give an identical offspring if we change the parents order.

	\item
		\begin{statement}
		For the EX operator why did we choose the vertex with the shortest edge-list to be the next vertex? Is it true, that if both parents have a common edge in their edge-list, then the offspring always has it in its edge list?
		\end{statement}

		We choose that vertex to finish with it as soon as possible, so that we do not leave open paths. \\
		The second statement is true. If both parents have it, it will be chosen in the offspring. We can see it in this part of the algorithm:
		\emph{If there is a common edge in the current vertex’s edge list, then that vertex will be in the next current vertex}
	\item
		\begin{statement}
		Write a program to solve the Traveling Salesman problem using a genetic algorithm. Use permutation representation, EX crossover and inversion mutation. The test-problem should be the one you designed on the last lecture.
		\end{statement}
		In my problem I have omitted where we have to check if there are common edges in the lists because in TSP the graph is complete (all edges are connected to each other).



\end{enumerate}


\begin{thebibliography}{9}

\bibitem{Course Webpage}
Course Webpage
\\\texttt{http://math.bme.hu/~safaro/evolalgen.html}


\bibitem{Webpage4}
\texttt{https://tex.stackexchange.com/}


\end{thebibliography}


\end{document}
